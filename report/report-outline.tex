\documentclass[headsepline,footsepline,footinclude=false,oneside,fontsize=10pt,paper=a4]{scrbook}
\renewcommand*\familydefault{\sfdefault}
\usepackage[T1]{fontenc}
\usepackage[utf8]{inputenc}
\usepackage{amsmath,amsfonts}
\usepackage{geometry}
\usepackage[french]{babel}
\usepackage{sectsty}
\geometry{hmargin=4.5cm, vmargin=4cm}
\usepackage{enumitem}

\sectionfont{\large\bfseries}


\begin{document}

\KOMAoptions{headsepline=false}

\begin{center}
    \LARGE \textbf{REPORT OUTLINE} \\[0.5cm]
    \normalsize \textit{Federated RLHF Pipeline for Personalized Learning with ScholéAI}
\end{center}

\vspace{1.5cm}

\section*{1. Introduction}
\begin{itemize}[leftmargin=1.5em]
    \item Motivations: personalization, scalability, and alignment in educational AI.
    \item Problem: aligning generative models with student preferences in federated environments.
    \item High-level contributions and project scope.
\end{itemize}

\section*{2. Related Work}
\begin{itemize}[leftmargin=1.5em]
    \item Federated RLHF and educational recommendation systems.
    \item Overview of DPO, PPO, and their limitations.
    \item Positioning of FedBiscuit.
\end{itemize}

\section*{3. FedBiscuit Architecture}
\begin{itemize}[leftmargin=1.5em]
    \item End-to-end system design: clients, central server, and training loop.
    \item Client-side preference optimization and server-side aggregation.
    \item Advantages and current limitations of the approach.
\end{itemize}

\section*{4. Synthetic Dataset: Generation and Augmentation}
\begin{itemize}[leftmargin=1.5em]
    \item Prompt design using a knowledge graph for structured scenario generation.
    \item Integration of learning styles and diverse student profiles.
    \item Heuristic construction of preference pairs for DPO training.
    \item Implementation of a new dataloader in FedBiscuit.
\end{itemize}

\section*{4bis. New Evaluation Component}
\begin{itemize}[leftmargin=1.5em]
    \item PPE Benchmark ?
\end{itemize}

\section*{5. Experiment}
\begin{itemize}[leftmargin=1.5em]
    \item TL;DR summarization task as a reference benchmark.
    \item Task: Assess the training dataset size required.
    \item Simulation design: client count, model size, number of rounds, hardware specs.
    \item Implementation details.
\end{itemize}

\section*{6. Results}
\begin{itemize}[leftmargin=1.5em]
    \item Quantitative results: accuracy, alignment score, and system performance.
    \item Visualization with graphs and tables.
    \item Qualitative analysis and interpretation.
\end{itemize}

\section*{7. Discussion}
\begin{itemize}[leftmargin=1.5em]
    \item Lessons learned and what worked well.
    \item Challenges in federated preference alignment (data distribution, model variance).
    \item Ideas for extending the work: personalization, real-world data, privacy constraints.
\end{itemize}

\section*{8. Conclusion and Future Work}
\begin{itemize}[leftmargin=1.5em]
    \item Recap of objectives and core contributions.
    \item Next steps: Deployment with real student data and system improvements.
\end{itemize}

%\section*{Appendix}
%\begin{itemize}[leftmargin=1.5em]
%    \item Additional figures, training logs, and metrics.
%    \item Prompt templates and pairing heuristics.
%    \item Config files and code snippets for reproducibility.
%\end{itemize}

\end{document}
