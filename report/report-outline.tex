\documentclass[headsepline,footsepline,footinclude=false,oneside,fontsize=10pt,paper=a4]{scrbook}
\renewcommand*\familydefault{\sfdefault}
\usepackage[T1]{fontenc}
\usepackage[utf8]{inputenc}
\usepackage{amsmath,amsfonts}
\usepackage{geometry}
\usepackage[french]{babel}
\usepackage{sectsty}
\geometry{hmargin=4.5cm, vmargin=4cm}
\usepackage{enumitem}

\sectionfont{\large\bfseries}


\begin{document}

\KOMAoptions{headsepline=false}

\begin{center}
    \LARGE \textbf{REPORT OUTLINE} \\[0.5cm]
    \normalsize \textit{Federated RLHF Pipeline for Personalized Learning with ScholéAI}
\end{center}

\vspace{1.5cm}

\section*{1. Introduction}
\begin{itemize}[leftmargin=1.5em]
    \item Motivations: personalization, scalability, and alignment in educational AI.
    \item Problem: aligning generative models with student preferences in federated environments.
    \item Importance of privacy in educational data handling.
    \item Potential impacts of federated learning approaches on personalized education.
    \item High-level contributions and project scope, emphasizing synthetic data generation and DPO training.
\end{itemize}

\section*{2. Related Work}
\begin{itemize}[leftmargin=1.5em]
    \item Federated RLHF and educational recommendation systems.
    \item Comparison of standard RLHF approaches (DPO, PPO).
    \item Limitations of PPO: high computational cost, instability in small datasets.
    \item Advantages of DPO: direct optimization of preference alignment, efficiency.
    \item Review of synthetic data generation techniques for personalized learning.
    \item Positioning of FedBiscuit as a novel combination of DPO and federated techniques in educational contexts.
\end{itemize}

\section*{3. FedBiscuit Architecture}
\begin{itemize}[leftmargin=1.5em]
    \item Detailed end-to-end federated system: client-server architecture, data flow, iterative training loop.
    \item Client-side operations: data preprocessing, local DPO model training.
    \item Server-side aggregation: model averaging, privacy preservation methods.
    \item Current limitations: scalability, communication overhead.
    \item Auto-J Evaluation component for automated performance benchmarking.
    \end{itemize}

\section*{4. Synthetic Dataset: Generation, Augmentation and Evaluation}
\begin{itemize}[leftmargin=1.5em]
    \item Integration of explicit and implicit student profile data.
    \item Use of streamlit webapp to allow user-friendly parameter tuning.
    \item Prompt engineering technics used, and knowledge graph (KG) for structured data generation.
    \item Approaches to tailoring scenarios for different learning modalities and student profile.
    \item Structure of the prompt templates.
    \item Learning curriculum preference pairs.
    \item Automatic and human annotation of preference pairs to evaluate quality of synthetic data.
    %\item Changes in dataloader for handling new synthetic datasets.
    \item Human .
\end{itemize}

\section*{4bis. New Evaluation Component}
\begin{itemize}[leftmargin=1.5em]
    \item Introduction and rationale behind the PPE Benchmark or similar standardized benchmarking methods.
    \item Description of evaluation metrics tailored for personalization and preference alignment.
    \item Methods to integrate qualitative feedback into evaluation results.
    \item Effectiveness and reliability analysis of the evaluation approach.
\end{itemize}

\section*{5. Experiment}
\begin{itemize}[leftmargin=1.5em]
    \item TL;DR summarization task as baseline for system validation.
    \item Objective: Identify optimal training dataset size and configuration.
    \item Simulation setup: client number, synthetic dataset distribution.
    \item Tested model architectures and size variations.
    \item Explored hyperparameters: federated rounds, learning rates, batch sizes.
    \item Detailed hardware specifications for reproducibility.
    \item Implementation of checkpoints and reproducibility protocols (e.g., random seeds).
\end{itemize}

\section*{6. Results}
\begin{itemize}[leftmargin=1.5em]
    \item Comprehensive quantitative metrics: accuracy, alignment scores, efficiency, convergence.
    \item Comparative analysis of simulation configurations and dataset sizes.
    \item Visualizations: graphs, heatmaps, and informative tables.
    \item In-depth qualitative analysis: student preference alignment, edge cases, anomalies.
    \item Interpretation of results in federated learning context and practical implications.
\end{itemize}

\section*{7. Discussion}
\begin{itemize}[leftmargin=1.5em]
    \item Reflections on technical decisions, effectiveness, and lessons learned.
    \item Analysis of federated preference alignment challenges: data heterogeneity, communication overhead, client dropout.
    \item Discussion on synthetic data efficacy compared to real-world data.
    \item Methods to enhance personalization effectiveness further.
    \item Future privacy enhancements (differential privacy, secure aggregation).
\end{itemize}

\section*{8. Conclusion and Future Work}
\begin{itemize}[leftmargin=1.5em]
    \item Summary of achieved goals and core contributions in federated RLHF for education.
    \item Potential next steps: deployment with real student data, model scalability improvements.
    \item Directions for future research: real-time feedback integration, adaptive learning paths.
    \item Importance of user-friendly interfaces for educators using FedBiscuit.
\end{itemize}

%\section*{Appendix}
%\begin{itemize}[leftmargin=1.5em]
%    \item Additional figures, training logs, and detailed metrics.
%    \item Prompt templates and pairing heuristics used.
%    \item Configuration files and reproducible code snippets.
%\end{itemize}

\end{document}
